\documentclass[12pt]{article}
\usepackage{amsmath}
\usepackage{amsthm}
\usepackage{amssymb}
\usepackage[left=2cm,top=1cm,right=3cm,bottom=1cm]{geometry}
\usepackage{graphicx}
\usepackage{listings}

\newtheorem{axiom}{Axiom}
\newtheorem{remark}{Remark}


\begin{document}

\begin{center}\LARGE\bf
    Sets, Relations and Functions
\end{center}

\section{The Axiom of Extension}
Sets (or collections) have elements (or members). Any object (concrete or otherwise), may be an element
of a set. This extends to sets, which may be an element of some other set. In fact, it is generally sufficient
to only discuss sets of sets, as the properties will extend to all other sets. There are some ways to relate sets: belonging, inclusion and equality.

If an object $x$ is an element of set $A$, we say that it \textit{belongs} to $A$, or "$x$ is contained
in $A$". Denote this as $x$ $\epsilon$ $A$.

If a set $A$ is \textit{included} in some set $B$, then $A$ is a subset of $B$. Denote this as $A \subset B$ or
$B \supset A$. Some related definitions: set inclusion is \textit{reflexive} since a set is included in
itself ($A \subset A$)\footnote{Clearly, $A \subset B$ is the same as $A \subseteq B$ in this notation.}; set inclusion is \textit{transistive} as $A \subset B$ and $B \subset C$ implies
$A \subset C$. (Conversely, belonging is neither reflexive or transistive).

Sets can also be related by equality ($A = B$), as defined by the Axiom of Extension.
\begin{axiom}[Axiom of Extension]
    Two sets are equal if and only if they have the same elements.
\end{axiom}
In terms of  inclusion, this can be rewritten as "$A \subset B$ and $B \subset A \Leftrightarrow A = B$".

A set is said to be a \textit{proper} subset of another if $A \subset B$ and $A \neq B$; that is, all
subsets are proper except from the set itslf, which is a subset but not proper.

Equality is \textit{symmetric}, that is $A = B \Leftrightarrow B = A$, whereas set inclusion is \textit{antisymmetric}
(that is, $A \subset B$ and $B \subset A \Rightarrow A = B$)\footnote{Antisymmetry is defined by "if $aRb$ and $bRa$, then $a = b$" for
some binary relation R.}.

As an aside, the Axiom of Extension implies that all elements of a set are distinct; that is to say, $\{1, 2, 2\} = \{1, 2\}$ (using notation
introduced later). There is no element $x$ that belongs to the left hand side for which $x$ belonging to the right hand side isn't
also true.

\section{The Axiom of Specification}
The basis of most Set Theoretic principles is to construct new sets from old sets. The most important such principle
is the so called "\textit{Aussonderungsaxiom}":
\begin{axiom}[Axiom of Specification]
    To every set A and every condition S(x) there correponds a set B whose elements are exactly those
    elements x of A for which S(x) holds.
\end{axiom}
That is to say, anything one can say about a the elements of a set defines a subset. The condition must form a
valid \textit{sentence}. A sentence is formed using the atomic sentences of belonging ($x$ $\epsilon$ $A$)
and equality ($A = B$), which are used to create more specific sentences using the following logical operators:
\begin{itemize}
    \item and
    \item or
    \item if and only if
    \item not
    \item if - then - (equivalently \textit{implies})
    \item for some (equivalently \textit{there exists})
    \item for all.
\end{itemize}
There are a few general rules for sentence construction\footnote{Some general notes: the set of a single object is not the same as that object itself;
"for some $y$ ($x$ $\epsilon$ $A$)" is equivalent to "$x$ $\epsilon$ $A$"; "for some $x$ ($x$ $\epsilon$ $A$)" and
"for some $y$ ($y$ $\epsilon$ $A$)" are equivalent.}:
\begin{enumerate}
    \item \textit{and}, \textit{or} and \textit{if and only if} are placed between two sentences
    \item \textit{not} can be placed before a sentence enclosed by parentheses
    \item the dashes in \textit{if - then - } are replaced with sentences
    \item follow \textit{for some} or \textit{for all} by a letter, in turn followed by a sentence
\end{enumerate}

We can write the Axiom of Specification symbolically as: $B = \{x \text{ }\epsilon \text{ }A: S(x)\}$. This set $B$ is uniquely
specified due to the Axiom of Extension.

\begin{remark}[Russell paradox]
    Nothing contains everything (or equivalently, there is no universe)
\end{remark}
\begin{proof}
    Assume there is a set $A$ which contains everything.

    Consider the condition: $\text{not } (x \text{ } \epsilon \text{ } x)$. Let us rewrite this as ($x$ $\epsilon'$ $x$) for
    ease of use.

    Constructing the set $B$ from $A$ where this condition holds:
    \begin{displaymath}
        B = \{x \text{ } \epsilon \text{ } A: x \text{ } \epsilon' \text{ } x\}
    \end{displaymath}

    Following from the Axiom of Extension,
    \begin{displaymath}
        y \text{ } \epsilon \text{ } B \text{ if and only if } (y \text{ } \epsilon \text{ } A \text{ and } y \text{ } \epsilon' \text{ } y)
    \end{displaymath}

    For $A$ to contain everything, this must hold for any $y$. Consider $y = B$.
    If $B$ $\epsilon$ $A$, then $B$ $\epsilon$ $B \Leftrightarrow B$ $\epsilon'$ $B$. This is clearly a
    contradiction so $A$ does not contain $B$ and by extension, $A$ cannot contain everything.
\end{proof}

The Russell Paradox gives us an example of a condition that produces an illegal set. Some texts name these
illegal sets as "classes".

\section{Unordered Pairs}
In order to make any progress it is necessary for us to make an assumption: \textit{there exists a set}.
This will be formulated more carefully later on but for now this is sufficient.

Given the assumption that there exists a set, one can easily construct the \textit{empty set} using the
Axiom of Specification, using the condition $x \neq x$. Clearly there are no
elements of such a set. This set is unique by the Axiom of Extension. Denote this as $\{x \text{ } \epsilon \text{ } A: x \neq x \} = \varnothing$.

For any set $A$, $\varnothing \subset A$ as every element in $\varnothing$ is in $A$. This is a vacuous truth.\footnote{The
arguement can also be made from the other direction: the only case where $\varnothing \subset A$ is false
is where there is some element in $\varnothing$ that is not contained in $A$. Since $\varnothing$ is empty, this
doesn't make any sense and so the statement can never be false.}

\begin{axiom}[Axiom of Pairing]
    For any two sets there exists a set that they both belong to.
\end{axiom}
That is, for some sets $a$ and $b$, there exists a set $A$ such that $a \text{ } \epsilon \text{ } A \text{ and } b \text{ }
\epsilon \text{ } A$.

Consider the set that contains $a$ and $b$ and nothing else\footnote{The existence of such a set is actually equivalent to the Axiom of Pairing:
if there is some set $A$ such that $a \; \epsilon \; A$ and $b \; \epsilon \; A$, then we can construct
a set with only $a$ and $b$ in using the sentence "$x = a \text{ or } x = b$".}.
This can be constructed from the Axiom of Specification using the condition $S(x) =
"x = a \text{ or } x = b"$. By the Axiom of Extension, this set is unique. We call this the \textit{pair}
or, more specifically, the \textit{unordered pair} formed by $a$ and $b$. Denote this set as
\begin{equation}
    \{x \; \epsilon \; A: x = a \text{ or } x = b\} = \{a, b\}.
\end{equation}
Note that this isn't the same as the set of all elements in $a$ or $b$ but the set containing only two
elements: $a$ and $b$.

The \textit{singleton} of $a$ is a special case unordered pair $\{a, a\}$, denoted by $\{a\}$, where the set has $a$
as its only element. $a \; \epsilon \; A$ is equivalent to $\{a\} \subset A$.

Considering the unordered pair of any set and the empty set, it follows from the Axiom of Pairing
that every set is an element of some other set. We can also infer the existance of infinitely many sets by construction:
first consider the singletons $\varnothing, \{\varnothing\}, \{\{\varnothing\}\}$ etc.; now consider the pairs of such sets, and the
pairs of these and the singletons and so on. Note that all of these sets are unique by the Axiom of Extension.

If $S(x)$ is a condition on $x$, then the set of all $x$ that satisfy this condition is given by $\{x:S(x)\}$,
granted that this is a valid set. Rewriting the generic set constructed by the Axiom of Specification,
\begin{displaymath}
    \{x\;\epsilon\;A: S(x)\} = \{x: x\;\epsilon\;A \text{ and } S(x)\}.
\end{displaymath}

\section{Unions and Intersections}
\begin{axiom}[Axiom of Unions]
    For every collection of sets there exists a set that contains all the elements that belong to at least
    one set of the given collection.
\end{axiom}
In other words, let $x\;\epsilon\;U$ only if $x\;\epsilon\;X$ for some $X$ in collection $\mathcal{C}$\footnote{Set and collection
can be used interchangably but both are used here to make it easier to discern what is being referred to.}. Currently, this set $U$
could contain elements not belonging to any of the sets in the collection. To construct a set that only contains elements belonging to some set in
$\mathcal{C}$, use the Axiom of Specification with the condition:
\begin{displaymath}
    \{x\;\epsilon\;U: x\;\epsilon\;X \text{ for some } X \text{ in } \mathcal{C}\}
\end{displaymath}
Change the notation and name this new set $U$. This set is the \textit{union} of the collection of sets (that
is the set of sets) $\mathcal{C}$, which is unique by the Axiom of Extension. Denote this as
\begin{equation}
    \bigcup\;\{X: X\;\epsilon\;\mathcal{C}\}.
\end{equation}

We shall now introduce various facts regarding unions (proofs are in Appendix \ref{Appendix: Unions and Intersections proofs}).
The facts regarding singletons are trivial.
\begin{remark}
    $\bigcup\;\{X: X\;\epsilon\;\varnothing\} = \varnothing$
\end{remark}

\begin{remark}
    $\bigcup\;\{X: X\;\epsilon\;\{A\}\} = A$
\end{remark}

There is special notation for the union of pairs of sets:
\begin{equation}
    \bigcup\;\{X: X\;\epsilon\;\{A, B\}\} = A \cup B.
\end{equation}

We may now introduce properties of pairs.

\begin{remark}
    $x\;\epsilon\;A \cup B \text{ if and only if } x\;\epsilon\;A \text{ or } x\;\epsilon\;B$
\end{remark}

Thus, we can write:
\begin{equation}
    A \cup B = \{x: x\;\epsilon\;A \text{ or } x\;\epsilon\;B\}.
\end{equation}

Using this, we prove the following elementary properties of the union of pairs:
\begin{remark}
    $A \cup\varnothing = A$
\end{remark}

\begin{remark}[Commutativity]
    $A \cup B = B \cup A$
\end{remark}

\begin{remark}[Associativity]
    $A \cup (B \cup C) = (A \cup B) \cup C$
\end{remark}

\begin{remark}[Idempotence]
    $A \cup A = A$
\end{remark}

\begin{remark}
    $A \subset B \text{ if and only if } A \cup B = B$
\end{remark}

\begin{remark}
    $\{a\} \cup \{b\} = \{a, b\}$
\end{remark}

Define $\{a, b, c\} = \{a\} \cup \{b\} \cup \{c\}$.

\begin{remark}
    $\{a, b, c\} = \{x: x = a \text{ or } x = b \text{ or } x = c\}$
\end{remark}

This remark means that, for every three sets, there exists a unique set that contains the three sets and
nothing else. This is the \textit{unordered triple} formed by them. Clearly, one can extend this arguement to \textit{quadruples}
and so on.

We develop a the topic of intersections in a way analagous to that of unions.
\begin{remark}
    For each non-empty collection of sets there exists a set that contains exactly those elements that
    belong to every set of the given collection.
\end{remark}
\begin{proof}
    We can prove this by simply constructing such a set for a general non-empty collection using the Axiom of Specification.
    Given that the collection $\mathcal{C}$ is non-empty, let $A$ be some set in $\mathcal{C}$.
    Thus we can write
    \begin{equation*}
        V = \{x\;\epsilon\;A: x\;\epsilon\;X \text{ for every } X\text{ in } \mathcal{C}\}.
    \end{equation*}
    Given that $A$ is included in the statement "for every $X$ in $\mathcal{C}$", we can in fact write:
    \begin{equation}
        V = \{x: x\;\epsilon\;X \text{ for every } X\text{ in } \mathcal{C}\}.
    \end{equation}
\end{proof}
The set $V$ is the \textit{intersection} of the collection $\mathcal{C}$, unique by the Axiom of Extension.
Denote this by:
\begin{equation}
    \bigcap\;\{X:X\;\epsilon\;\mathcal{C}\}
\end{equation}

The following notation is used to denote the intersection of sets $A$ and $B$ (that is, of a pair):
\begin{equation}
    A \cap B = \{x:x\;\epsilon\;A \text{ and } x\;\epsilon\;B\}
\end{equation}

Using this, we prove the following elementary properties of the intersection of pairs (proofs in Appendix \ref{Appendix: Unions and Intersections proofs}):
\begin{remark}
    $A \cap\varnothing = \varnothing$
\end{remark}

\begin{remark}[Commutativity]
    $A \cap B = B \cap A$
\end{remark}

\begin{remark}[Associativity]
    $A \cap (B \cap C) = (A \cap B) \cap C$
\end{remark}

\begin{remark}[Idempotence]
    $A \cap A = A$
\end{remark}

\begin{remark}
    $A \subset B \text{ if and only if } A \cap B = A$
\end{remark}

In the case where $A \cap B = \varnothing$, the pair of sets are called \textit{disjoint}. For a collection of sets,
if any two distinct sets in the collection are disjoint, the collection is called \textit{pairwise disjoint}.

We shall now prove the \textit{distributive laws}. As with many Set Theoretic proofs, to prove $A = B$, we show $A \subset B$ and $A \supset B$.

\begin{remark}
    $A \cap (B \cup C) = (A \cap B) \cup (A \cap C)$
\end{remark}
\begin{proof}
    If $x$ belongs to the left side, then $x$ belongs to $A$ and to either $B$ or $C$. If $x$ is in $B$,
    then $x$ is in $A \cap B$ and, if $x$ is in $C$, then $x$ is in $A \cap C$. Thus, $x$ is in either $A \cap B$
    or $A \cap C$ and so is in the right side in either case.

    If $x$ belongs to the right side, it belongs to either $A$ and $B$ or to $A$ and $C$. In both cases,
    it belongs to $A$. It either belongs to $B$ or $C$ so $x$ belongs to $B \cup C$. Thus, $x$ is also in the left side
    in either case.

    This is true for all $x$ and so we can conclude equality by the Axiom of Extension.
\end{proof}

\begin{remark}
    $A \cup (B \cap C) = (A \cup B) \cap (A \cup C)$
\end{remark}
\begin{proof}
    If $x$ belongs to the left side, then $x$ belongs either to $A$ or to both $B$ and $C$. If $x$ is in $A$,
    then $x$ is in both $A \cup B$ and $A \cup C$. If $x$ is in both $B$ and $C$, then $x$ is, once again, in both
    $A \cup B$ and $A \cup C$. Thus, in any case, $x$ belongs to the right side. Thus the right side includes the left.

    The reverse inclusion is proved by observing that if $x$ belongs to both $A \cup B$ and $A \cup C$, then
    $x$ belongs either to $A$ or to both $B$ and $C$.
\end{proof}

Here is a further proof to provide more examples of Set Theoretic proofs.

\begin{remark}
    A necessary and sufficient condition that $(A \cap B) \cup C = A \cap (B \cup C)$ is that $C \subset A$.
    Also, observe that the condition has nothing to do with the set $B$.
\end{remark}
\begin{proof}
    First, we prove that $C \subset A$ implies $(A \cap B) \cup C = A \cap (B \cup C)$.

    If $C \subset A$, then $x \in C$ implies that $x \in A$.

    Assume that $x \in C$. If $x \in B$, then $x \in (A \cap B)$ and $x \in C$. Thus, $x \in (A \cap B) \cup C$.
    Also, $x \in A$ and $x \in C$ so $x \in A \cap (B \cup C)$.

    If $x \notin B$, then $x \notin A \cap B$ but $x \in C$ so $x \in (A \cap B) \cup C$.
    Also, $x \in A$ and $x \in B \cup C$ so $x \in A \cap(B \cup C)$.

    This is true for all cases of $x$ so by the Axiom of Extension, $(A \cap B) \cup C = A \cap (B \cup C)$.
    This is true independent of the set $B$.

    Next, we prove that $(A \cap B) \cup C = A \cap (B \cup C)$ implies $C \subset A$.

    Assume $(A \cap B) \cup C = A \cap (B \cup C)$.

    If $x \in (A \cap B) \cup C$, either both $x \in A$ and $x \in B$ or $x \in C$.
    If $A \cap (B \cup C)$, either both $x \in A$ and $x \in B$ or both $x \in A$ and $x \in C$.

    Thus, if $x \in C$, for the right side to be true, $x \in A$ must also be true. Thus, $x \in C$ implies $x \in A$
    so $C \subset A$.

    Once again, this argument is independent of the set $B$.
\end{proof}

\section{Complements and Powers}
The \textit{difference} between sets $A$ and $B$, or \textit{relative complement} of $B$ in $A$, is
defined by
\begin{equation}
    A - B = \{x: x \in A \text{and} x \notin B\}.
\end{equation}

Consider some set $E$ and its collection of subsets. We can then consider the complement of one of these
subsets (with respect to $E$) as the absolute complement. Denote the (temporarily) absolute complement of $A$
as $A'$. With this notation, we can state various basic facts (proofs in Appendix \ref{Appendix: Complements and Powers proofs}):

\begin{remark}
    $(A')' = A$
\end{remark}

\begin{remark}
    $A \subset B$ if and only if $B' \subset A'$
\end{remark}

The following properties are known as the \textit{De Morgan laws}:
\begin{equation}
    (A \cup B)' = A' \cap B'
\end{equation}
\begin{equation}
    (A \cap B)' = A' \cup B'
\end{equation}
It is common for set theory properties to come in pairs. That is, replacing every set by its complement,
reversing inclusions and swapping unions and intersetions produces another theorem. This is known as the \textit{principle
of duality}.

\begin{remark}
    $A - B = A \cap B'$
\end{remark}
\begin{proof}
    \begin{eqnarray*}
        A - B &=\{x: x \in A \text{ and } x \notin B\}\\
              &= \{x: x \in A \text{ and } x \in B'\}\\
              &= A \cap B'
    \end{eqnarray*}
\end{proof}

The \textit{symmetric difference} or \textit{Boolean sum} of two sets, $A$ and $B$, is
defined by $A + B = (A - B) \cup (B - A)$. This is both commutative and associative.

Let us now formalise the set of all subsets of some set:
\begin{axiom}[Axiom of powers]
    For each set there exists a collection of sets that contains among its elements all the subsets
    of the given set.
\end{axiom}
Note that collection has been used instead of set to aid clarity. Symbolically, if $E$ is some set, there
exists some set $\mathcal{P}$ such that if $X \subset E$, then $X \in \mathcal{P}$. In order to construct
a set that \textit{only} contains the subsets of a given set, apply the axiom of specification with the
condition that $X \subset E$. That is, the set is $\{X \in \mathcal{P}: X \subset E\}$. Given that any
set that satisfies the condition also satisfies $X \in \mathcal{P}$ by definition, change the notation
such that $\mathcal{P} = \{X: X \subset E\}$. This set is called the \textit{power set}, more specifically
notated as $\mathcal{P}(E)$ to highlight that it contains all the subsets of $E$.

%%%%%%%%%%%%%%%%%%%%%%%%%%%%%%%%%%%%%%%%%%%%%%%%%%%%%%%%%%%%%%%%%%%%%%%%%%%%%%%%%%%%%%%%%%%%%%%%%%%%


%%%%%%%%%%%%%%%%%%%%%%%%%%%%%%%%%%%%%%%%%%%%%%%%%%%%%%%%%%%%%%%%%%%%%%%%%%%%%%%%%%%%%%%%%%%%%%%%%%%%


\appendix
\section{Proofs of Union and Intersection properties}\label{Appendix: Unions and Intersections proofs}
\begin{remark}
    $\bigcup\;\{X: X\;\epsilon\;\varnothing\} = \varnothing$
\end{remark}
\begin{proof}
    $\bigcup\;\{X: X\;\epsilon\;\varnothing\}$ is the set of all elements of the sets that are, in turn,
    elements of $\varnothing$. This set has no elements, so by extension the set of these non-existant
    sets is also empty.
\end{proof}

\begin{remark}
    $\bigcup\;\{X: X\;\epsilon\;\{A\}\} = A$
\end{remark}
\begin{proof}
    Similarly, $\bigcup\;\{X: X\;\epsilon\;\{A\}\}$ is the set of all elements in each set in the
    collection $\{A\}$. As this is a singleton, this is just the set $A$, by definition.
\end{proof}


\begin{remark}
    $x\;\epsilon\;A \cup B \text{ if and only if } x\;\epsilon\;A \text{ or } x\;\epsilon\;B$
\end{remark}
\begin{proof}
    If $x\;\epsilon\;A \text{ or } x\;\epsilon\;B$, by definition $x\;\epsilon\;\{A, B\}$ so $x\;\epsilon\;A \cup B$
    immediately follows.

    If $x\;\epsilon\;A \cup B$, then the definition of the union states that $x\;\epsilon\;X$ for some
    $X \text{ in } \{A, B\}$. This is equivalent to saying $x\;\epsilon\;A \text{ or } x\;\epsilon\;B$.
\end{proof}

Thus, we can write:
\begin{equation}
    A \cup B = \{x: x\;\epsilon\;A \text{ or } x\;\epsilon\;B\}.
\end{equation}

Using this, we prove the following elementary properties of the union of pairs:
\begin{remark}
    $A \cup\varnothing = A$
\end{remark}
\begin{proof}
    \begin{displaymath}
        A \cup \varnothing = \{x: x\;\epsilon\;A \text{ or } x\;\epsilon\;\varnothing\}
    \end{displaymath}
    $x\;\epsilon\;\varnothing$ is always false so
    \begin{displaymath}
        A \cup \varnothing = \{x: x\;\epsilon\;A\} = A
    \end{displaymath}
\end{proof}

\begin{remark}[Commutativity]
    $A \cup B = B \cup A$
\end{remark}
\begin{proof}
    \begin{align*}
        A \cup B &= \{x: x\;\epsilon\;A \text{ or } x\;\epsilon\;B\}\\
                 &= \{x: x\;\epsilon\;B \text{ or } x\;\epsilon\;A\}\\
                 &= B \cup A
    \end{align*}
    where we use the fact that the logical operator \textit{or} is commutative.
\end{proof}

\begin{remark}[Associativity]
    $A \cup (B \cup C) = (A \cup B) \cup C$
\end{remark}
\begin{proof}
    \begin{align*}
        A \cup (B \cup C) &= A \cup (\{x: x\;\epsilon\;B \text{ or } x\;\epsilon\;C\})\\
                          &= \{x: x\;\epsilon\;A \text{ or } (x\;\epsilon\;B \text{ or } x\;\epsilon\;C)\}\\
                          &= \{x: (x\;\epsilon\;A \text{ or } x\;\epsilon\;B) \text{ or } x\;\epsilon\;C\}\\
                          &= (A \cup B) \cup C
    \end{align*}
    where we used the fact that the logical operator \textit{or} is associative.
\end{proof}

\begin{remark}[Idempotence]
    $A \cup A = A$
\end{remark}
\begin{proof}
    \begin{align*}
        A \cup A &= \{x: x\;\epsilon\;A \text{ or } x\;\epsilon\;A\}\\
                 &= \{x: x\;\epsilon\;A\}\\\
                 &= A
    \end{align*}
\end{proof}

\begin{remark}
    $A \subset B \text{ if and only if } A \cup B = B$
\end{remark}
\begin{proof}
    If $A \cup B = \{x: x\;\epsilon\;A\text{ or } x\;\epsilon\;B\} = B$, then there must be no case
    where $x\;\epsilon\;A$ where $x\;\epsilon\;B$ is not also true.
    Thus, any element $x\;\epsilon\;A$ must also be in $B$, so $A \subset B$ by definition.

    If $A\subset B$, every element in $A$ is also in $B$. Thus,
    \begin{align*}
        A \cup B &= \{x: x\;\epsilon\;A\text{ or } x\;\epsilon\;B\}\\
                 &= \{x: x\;\epsilon\;B\}\\
                 &= B.
    \end{align*}
\end{proof}

\begin{remark}
    $\{a\} \cup \{b\} = \{a, b\}$
\end{remark}
\begin{proof}
    \begin{align*}
        \{a\} \cup \{b\} &= \{x: x\;\epsilon\;\{a\} \text{ or } x\;\epsilon\;\{b\}\}\\
                         &= \{x: x = a \text{ or } x = b\}\\
                         &=\{a, b\}
    \end{align*}
\end{proof}

Define $\{a, b, c\} = \{a\} \cup \{b\} \cup \{c\}$.

\begin{remark}
    $\{a, b, c\} = \{x: x = a \text{ or } x = b \text{ or } x = c\}$
\end{remark}
\begin{proof}
    \begin{align*}
        \{a, b, c\} &= \{a\} \cup \{b\} \cup \{c\}\\
                    &= \{x: x = a \text{ or } x = b\} \cup \{c\}\\
                    &= \{x: x\;\epsilon\;\{x: x = a \text{ or } x = b\} \text{ or } x\;\epsilon\;\{c\}\}\\
                    &= \{x: x = a \text{ or } x = b \text{ or } x = c\}.
    \end{align*}
\end{proof}

%%%%%%%%%%%%%%%%%%%%%%%%%%%%%%%%%%%%%%%%%%%%%%%%%%%%%%%%%%%%%%%%%%%%%%%%%%%%%%%%%
The following are the proofs for the analagous properties of intersections of pairs.

\begin{remark}
    $A \cap\varnothing = \varnothing$
\end{remark}
\begin{proof}
    \begin{equation*}
        A \cap \varnothing = \{x: x\;\epsilon\;A \text{ and } x\;\epsilon\;\varnothing\}
    \end{equation*}
    $x\;\epsilon\;\varnothing$ is always false so there are no elements in this set. Thus,
    $A \cap\varnothing = \varnothing$
\end{proof}

\begin{remark}[Commutativity]
    $A \cap B = B \cap A$
\end{remark}
\begin{proof}
    \begin{align*}
        A \cap B &= \{x: x\;\epsilon\;A \text{ and } x\;\epsilon\;B\}\\
                 &= \{x: x\;\epsilon\;B \text{ and } x\;\epsilon\;A\}\\
                 &= B \cap A
    \end{align*}
    where we use the fact that the logical operator \textit{and} is commutative.
\end{proof}

\begin{remark}[Associativity]
    $A \cap (B \cap C) = (A \cap B) \cap C$
\end{remark}
\begin{proof}
    \begin{align*}
        A \cap (B \cap C) &= A \cap (\{x: x\;\epsilon\;B \text{ and } x\;\epsilon\;C\})\\
                          &= \{x: x\;\epsilon\;A \text{ and } (x\;\epsilon\;B \text{ and } x\;\epsilon\;C)\}\\
                          &= \{x: (x\;\epsilon\;A \text{ and } x\;\epsilon\;B) \text{ and } x\;\epsilon\;C\}\\
                          &= (A \cap B) \cap C
    \end{align*}
    where we used the fact that the logical operator \textit{and} is associative.
\end{proof}

\begin{remark}[Idempotence]
    $A \cap A = A$
\end{remark}
\begin{proof}
    \begin{align*}
        A \cap A &= \{x: x\;\epsilon\;A \text{ and } x\;\epsilon\;A\}\\
                 &= \{x: x\;\epsilon\;A\}\\
                 &= A
    \end{align*}
\end{proof}

\begin{remark}
    $A \subset B \text{ if and only if } A \cap B = A$
\end{remark}
\begin{proof}
    If $A \cap B = \{x: x\;\epsilon\;A\text{ and } x\;\epsilon\;B\} = A$, then every element
    which belongs to $A$ must also belong to $B$, as $x\;\epsilon\;B$ must be true for all such elements.
    Thus, any element $x\;\epsilon\;A$ must also be in $B$, so $A \subset B$ by definition.

    Another way to see this is that $A \cap B \subset B$ by definition. Thus, by the initial assumption, $A \subset B$.

    If $A\subset B$, every element for all $x\;\epsilon\;A$, $x\;\epsilon\;B$ is also true. Thus,
    \begin{align*}
        A \cap B &= \{x: x\;\epsilon\;A\text{ and } x\;\epsilon\;B\}\\
                 &= \{x: x\;\epsilon\;A\}\\
                 &= A.
    \end{align*}
\end{proof}

\section{Proofs of Complements and Powers properties}\label{Appendix: Complements and Powers proofs}
\begin{remark}
    $A \subset B$ if and only if $B' \subset A'$
\end{remark}
\begin{proof}
    If $x \in B' \implies x \in A'$, then there is no case where $x$ isn't in $B$ where it is in $A$.
    Thus, if $x \in A$, it must be in $B$ also so $x \in A \implies x \in B$.

    For the reverse direction, if $x \in A \implies x \in B$ there is no case where $x$ is in $A$ where it isn't also
    in $B$. Thus, if $x$ isn't in $B$, it can't be in $A$ either. Thus, $x \in B' \implies x \in A'$.
\end{proof}

The following are the proofs of the De Morgan Laws:
\begin{remark}
    $(A \cup B)' = A' \cap B'$
\end{remark}
\begin{proof}
    We shall prove this by showing that each side is a subset of the other side respectively.

    Let $x \in (A \cup B)'$. That is, $x \notin A \cup B$. $A \cup B = \{y: y \in A \text{ or } y \in B\}$
    so both $x \notin A$ and $x \notin B$. Thus, $x \in A'$ and $x \in B'$, which implies that
    $x \in A' \cap B'$. Thus, $(A \cup B)' \subset A' \cap B'$.

    Let $x \in A' \cap B'$. Assume that $x \in A \cup B$. $x \in A' \cap B' \implies x \notin A \text{ and }
    x \notin B$ are both true. Thus there is no $x$ for which $x \in A$ or $x \in B$ and so we have found
    a contradiction. Thus $x \in (A \cup B)'$ and we can conlude $(A \cup B)' \supset A' \cap B'$
\end{proof}

\begin{remark}
    $(A \cap B)' = A' \cup B'$
\end{remark}
\begin{proof}
    We shall prove this by showing that each side is a subset of the other side respectively.

    Let $x \in (A \cap B)'$. That is, $x \notin A \cap B$. $A \cap B = \{y: y \in A \text{ and } y \in B\}$
    so either $x \notin A$ or $x \notin B$ or both, which implies that
    $x \in A' \cup B'$ (as $A' \cup B' = \{x: x \in A' \text{ or } x \in B'$\}). Thus, $(A \cap B)' \subset A' \cup B'$.

    Let $x \in A' \cup B'$. Assume that $x \in A \cap B$. $x \in A' \cup B' \implies x \notin A$ and
    $x \notin B$, which contradicts the assumption. Thus $x \in (A \cap B)'$ and we can conlude $(A \cap B)' \supset A' \cup B'$
\end{proof}

\begin{remark}[Commutativity]
    $A + B = B + A$
\end{remark}
\begin{proof}
    \begin{align*}
        A + B &= (A - B) \cup (B - A)\\
              &= \{x: x \in (A - B) \text{ or } x \in (B - A)\}\\
              &= \{x: x \in (B - A) \text{ or } x \in (A - B)\}\\
              &= B + A
    \end{align*}
\end{proof}

\begin{remark}[Associativity]
    $A + (B + C) = (A + B) + C$
\end{remark}
\begin{proof}
    We shall prove this by considering every case where some $x$ belongs to some combination of $A, B$
    and $C$. There are eight such cases (draw a Venn diagram to convince yourself that these are all the cases):
    \begin{itemize}
        \item $x \in A, x \in B, x \in C$
        \item $x \in A, x \in B, x \notin C$
        \item $x \in A, x \notin B, x \in C$
        \item $x \notin A, x \in B, x \in C$
        \item $x \in A, x \notin B, x \notin C$
        \item $x \notin A, x \in B, x \notin C$
        \item $x \notin A, x \notin B, x \in C$
        \item $x \notin A, x \notin B, x \notin C$
    \end{itemize}
    If we show associativity holds for each case exhaustively, then it must hold generally.

    Consider $x \in A, x \in B, x \in C$.
    $(B + C) = (B - C) \cup (C - B) = (B \cap C')\cup (C \cap B')$
    We know that $x \notin C'$ so $x \notin B \cap C'$. Similarly, we know that
    $x \notin B'$ so $x \notin C \cap B'$. Thus $x \notin (B + C)$.

    Let $D = (B + C)$. Thus, $A + (B + C) = A + D = (A \cap D')\cup (D \cap A')$.
    From above, $x \notin D$ so $x \in (A \cap D')$ and thus $x \in (A \cap D')\cup (D \cap A')$.
    Therefore, $x \in A + (B + C)$

    We can make a similar argument to show that $x \in (A + B) + C$ but given the symmetry of this case
    (note that symmetric difference is commutative), we can immediately conclude this (see this by rewriting
    $A + (B + C) = (C + B) + A$).

    Now consider $x \notin A, x \in B, x \in C$.

    We know that $x \notin C'$ so $x \notin B \cap C'$. Similarly, we know that
    $x \notin B'$ so $x \notin C \cap B'$. Thus $x \notin (B + C)$ (or $x \notin D$).

    $x \notin A \implies x \notin A \cap D'$. Similarly, $x \notin D \implies x \notin D \cap A'$.
    Thus, $x \notin A + D$ so $x \notin A + (B + C)$.

    We know that $x \in A'$ so $x \in B \cap A'$. Thus $x \in (A + B)$.

    Let $E = A + B$, such that $E + C = (E \cap C') \cup (C \cap E')$. We know that $x \notin C'$ so
    $x \notin E \cap C'$. Similarly, $x \in (A + B) \implies x \in E$ (or equivalently, $x \notin E'$)
    so $x \notin (C \cap E')$. Thus $x \notin E + C$ so $x \notin (A + B) + C$.

    We can consider the rest of the cases by similar arguments (you should do this to complete the proof). In doing so, every value of
    $x$ is exhaustively considered. There is no element belonging to $A + (B + C)$ that does not also
    belong to $(A + B) + C$ and vice versa.

    Thus $A + (B + C) = (A + B) + C$ by the axiom of extension.
\end{proof}


\end{document}
