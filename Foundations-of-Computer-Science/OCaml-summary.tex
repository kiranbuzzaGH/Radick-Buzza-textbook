\documentclass[12pt]{article}
\usepackage{amsmath}
\usepackage[left=2cm,top=1cm,right=3cm,bottom=1cm]{geometry}
\usepackage{graphicx}
\usepackage{listings}


\begin{document}

\begin{center}\LARGE\bf
    Summary
\end{center}

Integers \texttt{min_int ... -3 -2 -1 0 1 2 3 ... max_int} are of type \textbf{int}. Booleans \texttt{true}
and \texttt{false} are of type \textbf{bool}. Characters of type \textbf{char} are like \texttt{'X'} and
\texttt{'!'}. Mathematical operators \texttt{+ - * / mod} take two integers and give another. Operators
\texttt{= < <= > >= <>} compare two values and evaluate to either \texttt{true} or \texttt{false}.
The conditional \texttt{if \textit{expression1} then \textit{expression2} else \textit{expression3}}
where \textit{expression1} has type \textbf{bool} and \textit{expression2} and \textit{expression3} have the same type as one another.
The boolean operators \texttt{&&} and \texttt{||} allow us to build compound boolean expressions.

Assigning a name to the result of evaluating

%%%%%%%%%%%%%%%%%%%%%%%%%%%%%%%%%%%%%%%%%%%%%%%%%%%%%%%%%%%%%%%%%%%%%%%%%%%%%%%%%

Remember to include parentheses around expressions used as function arguements.
For example h::t as an arguement will throw an error, whereas (h::t) will not.

\end{document}
