\documentclass[12pt]{article}
\usepackage{amsmath}
\usepackage[left=2cm,top=1cm,right=3cm,bottom=1cm]{geometry}
\usepackage{graphicx}
\usepackage{listings}


\begin{document}

\begin{center}\LARGE\bf
    Dictionaries and Functional Arrays
\end{center}

\section{Dictionaries}
A \textit{dictionary} associates some unique \textit{keys} with corresponding \textit{values}. In OCaml,
these keys and their values are stored in a list of \textit{pairs}. A pair is a special case of a \textit{tuple}, which
may contain two or more objects (not necessarily of the same type). For example, the tuple \texttt{(1, false, 'a')} has
type \texttt{int $\times$ bool $\times$ char}. We now present some useful functions when working with dictionaries.
\begin{lstlisting}
    add : 'a -> 'b -> ('a * 'b) list
    mklists : ('a * 'b) list -> 'a list * 'b list

    let rec add k v d =
      match d with
        [] -> [(k, v)]
      | (k', v')::t ->
          if k = k'
            then (k, v) :: t
            else (k', v') :: add k v t

    let rec mklists l =
      match l with
        [] -> ([], [])
      | (k, v)::t ->
          let (kt, vt) = mklists t in
            (k :: ks, k :: vs)
\end{lstlisting}
Note that the add function replaces a duplicate key with the newer key value pair. The mklists (make lists)
function decomposes a dictionary into a tuple of lists, one containing the keys, the other containing the values.
These can be extracted by pattern matching in the usual way. Alternatively, the following syntax may be used:
\begin{lstlisting}
    let first (x, _) = x
    let second (_, y) = y
\end{lstlisting}

\end{document}
