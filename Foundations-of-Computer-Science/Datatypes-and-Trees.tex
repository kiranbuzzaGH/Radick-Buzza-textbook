\documentclass[12pt]{article}
\usepackage{amsmath}
\usepackage[left=2cm,top=1cm,right=3cm,bottom=1cm]{geometry}
\usepackage{graphicx}
\usepackage{listings}


\begin{document}

\begin{center}\LARGE\bf
    Datatypes and Trees
\end{center}

\section{Exceptions}
In OCaml, \textit{run-time errors} are reported with \textit{exceptions}. An exception can be defined
for later use, with the syntax: \textbf{exception} \textit{name}. The exception can also accept arguments
of some specified type, using the syntax: \textbf{exception} \textit{name} \textbf{of} \textit{type}.
Such an exception is called with (\textit{name argument}). Raise an exception with the keyword \textbf{raise}. See
examples in the functions "take" and "drop", defined in the lists chapter.

It is also possible to \textit{handle} an exception with an \textit{exception handler}. Note that the types must be
consistent with the type of the function.
\begin{lstlisting}
    safe_divide : int -> int -> int

    let safe_divide x y =
      try x / y with
        Division_by_zero -> 0
\end{lstlisting}

\end{document}
