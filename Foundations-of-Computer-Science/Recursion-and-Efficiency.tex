\documentclass[12pt]{article}
\usepackage{amsmath}
\usepackage[left=2cm,top=1cm,right=3cm,bottom=1cm]{geometry}
\usepackage{graphicx}
\usepackage{listings}

\begin{document}

\begin{center}\LARGE\bf
    Recursion and Efficiency
\end{center}

\section{Recursive Functions}
A \textit{recursive} function is one which calls itself. This is an incredibly powerful concept in
functional languages. The syntax for this is: \texttt{let rec \textit{name arguement1 arguement2} ... = \textit{expression}}.
When definining a recursive function, it is important that you set a base case such that the function
doesn't continue to call itself with no end. For example:
\begin{lstlisting}
factorial : int -> int

let rec factorial x =
  if x < 0 then 0 else
    if x = 0 then 1 else
      x * factorial (x - 1)
\end{lstlisting}

\end{document}
