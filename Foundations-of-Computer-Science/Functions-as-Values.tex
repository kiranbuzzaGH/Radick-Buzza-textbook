\documentclass[12pt]{article}
\usepackage{amsmath}
\usepackage[left=2cm,top=1cm,right=3cm,bottom=1cm]{geometry}
\usepackage{graphicx}
\usepackage{listings}


\begin{document}

\begin{center}\LARGE\bf
    Functions as Values
\end{center}

When making larger programs, it is necessary to reuse other programs. There are many ways of doing this.

It is possible to pass a function to another function as an argument. For example, the function \texttt{map}
below applies function f to every element in list l.
\begin{lstlisting}
    map : ('a -> 'b) -> 'a list -> 'b list

    let rec map f l =
      match l with
        [] -> []
      | h::t -> f h :: map f t
\end{lstlisting}

It is also possible to use a function within another function without assigning a name in the namespace to the argument function.
We call this type of function an \textit{anonymous function}. The syntax is
\texttt{fun name -> \textit{expression}}. This can be used when a function is only applied in one place and is
short. For example, below we define a function that returns true for an element if it's even.
\begin{lstlisting}
    evens : int list -> bool list

    let evens l =
      map (fun x -> x mod 2 = 0) l
\end{lstlisting}

OCaml allows you to convert an operator into a function with the syntax \texttt{( \textit{operator} )}. For example,
\texttt{( <= ) 3 4} has the value \texttt{true} and \texttt{( + ) 3 4} has the value \texttt{7}.

\section{Partial Application}
In reality, functions that take multiple arguments are really multiple single argument functions applied to
each other. That is, a function \text{f a b} has type $\alpha \rightarrow \beta \rightarrow \gamma$, which we can write as
$\alpha \rightarrow (\beta \rightarrow \gamma)$. Thus it takes an argument of type $\alpha$ and returns a function of type
$\beta \rightarrow \gamma$, which in turn takes an argument of $\beta$ and returns a value of type $\gamma$. This can be written
explicitly as \texttt{let f = fun a -> fun b -> ...} instead of just \texttt{let f = ...}

As a result, it is possible to apply fewer than the total number of arguments to a function. This is called \textit{partial
application}. We now introduce various examples to demonstrate this.

The simplest example is \texttt{let add x y = x + y}. If we write \texttt{let f = add 6}, f is now a function that
adds 6. Notice that \texttt{add} has type int $\rightarrow$ int $\rightarrow$ int and f (and by extension \texttt{add 6}) has type int $\rightarrow$ int.

Recall that it is possible to produce a function from an operator with the syntax \texttt{( / )}. This fact can be
used in conjunction with partial application to create, for example, a function which returns 2 divided by its input:
\texttt{let fun = ( / ) 2}. This can be applied to all elements of a list using map.

Consider a function that maps a function over a list of lists:
\begin{lstlisting}
  mapl : ('a -> 'b) -> 'a list list -> 'b list list

  let rec mapl f l =
    match l with
      [] -> []
    | h::t -> map f h :: mapl f t
\end{lstlisting}
This can be rewritten using partial application: \texttt{let mapl f l = map (map f) l} or even \texttt{let
mapl f = map (map f)}. In this case, \texttt{map (map f)} has type $\alpha$ list list $\rightarrow \beta$ list list.

Recall the member function, which checks if some value is in a given list:
\begin{lstlisting}
  member : 'a -> 'a list -> bool

  let rec member n l =
    match l with
      [] -> false
    | h::t -> n = h || member n t
\end{lstlisting}
We want to write a similar function that checks if some element is in every list in a list of lists.
We can solve this using partial application of member. \texttt{member x} will have type $\alpha$ list $\rightarrow$ bool
so \texttt{map (member x)} has type $\alpha$ list list $\rightarrow$ bool list. Thus,
\begin{lstlisting}
  member_all : 'a -> 'a list list -> bool

  let member_all x ls =
    let booleans = map (member x) ls in
      not (member false booleans)
\end{lstlisting}.

If we wanted to make a function which shortens all lists in a list of lists by a given length, we can do the following:
\begin{lstlisting}
  new_len : int -> 'a list -> 'a list
  truncate : int -> 'a list list -> 'a list list

  let new_len n l =
    try take n l with
      Invalid_argument _ -> l

  let truncate n l =
    map (new_len n) l
\end{lstlisting}
Note that when handling the exception above, we have used the wildcard $\_$ rather than specifying a string so that
our code is more robust and maintainable. If your code depends on the string argument of an exception for error handling,
you should refactor your code to have more specific exceptions.

\end{document}
