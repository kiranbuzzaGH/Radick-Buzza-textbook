\documentclass[12pt]{article}
\usepackage{amsmath}
\usepackage[left=2cm,top=1cm,right=3cm,bottom=1cm]{geometry}
\usepackage{graphicx}
\usepackage{listings}


\begin{document}

\begin{center}\LARGE\bf
    Functions as Values
\end{center}

When making larger programs, it is necessary to reuse other programs. There are many ways of doing this.

It is possible to pass a function to another function as an argument. For example, the function \texttt{map}
below applies function f to every element in list l.
\begin{lstlisting}
    map : ('a -> 'b) -> 'a list -> 'b list

    let rec map f l =
      match l with
        [] -> []
      | h::t -> f h :: map f t
\end{lstlisting}

It is also possible to use a function within another function without assigning a name in the namespace to the argument function.
We call this type of function an \textit{anonymous function}. The syntax is
\texttt{fun name -> \textit{expression}}. This can be used when a function is only applied in one place and is
short. For example, below we define a function that returns true for an element if it's even.
\begin{lstlisting}
    evens : int list -> bool list

    let evens l =
      map (fun x -> x mod 2 = 0) l
\end{lstlisting}

OCaml allows you to convert an operator into a function with the syntax \texttt{( \textit{operator} )}. For example,
\texttt{( <= ) 3 4} has the value \texttt{true} and \texttt{( + ) 3 4} has the value \texttt{7}.
\end{document}
