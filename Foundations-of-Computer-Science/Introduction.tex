\documentclass[12pt]{article}
\usepackage{amsmath}
\usepackage[left=2cm,top=1cm,right=3cm,bottom=1cm]{geometry}
\usepackage{graphicx}
\usepackage{listings}


\begin{document}

\begin{center}\LARGE\bf
    Introduction
\end{center}

This text uses OCaml to teach programming fundamentals.

Once installed, one can use OCaml in the terminal by typing OCaml and writing your expression
after the \texttt{\#}. An expression should be ended with \texttt{;;}. If you use OCaml while in a
directory containing a *.ml file, you can use the functions contained in this file with the syntax:
\texttt{# #use "\textit{filename}.ml";;}.

The simplest way to leave OCaml is to use \texttt{exit 0;;}. One can interrupt a running program by
pressing Ctrl-C.

\section{Concepts}
An OCaml program is an \textit{expression} (or a collection of expressions) with a \textit{type}.
This expression can be evaluated to give a \textit{value}. Operators can be applied to sub-expressions
to yield larger expressions. Some operators only apply to certain types.

The type \textbf{int} is the integers. Your machine will have a so called \texttt{min\_int} and \texttt{max\_int}, which
are the smallest and largest integer that is available, respectively. Adding to \texttt{max\_int} will
wrap back around to \texttt{min\_int} as \texttt{min\_int = max\_int + 1}. A similar situation occurs when
subtracting from \texttt{min\_int}. Thus, the integers from \texttt{min\_int, ..., -1, 0, 1, ..., max\_int} is
are \textit{all} of the objects with type \textbf{int}. It is worth being careful when working close to the edges of this range
to prevent unexpected results.

The type \textbf{bool} is the Boolean values \texttt{true} and \texttt{false}. It is useful to have
a separate type for Booleans rather than using, say, \texttt{0} and \texttt{1} as this would introduce
potential situations where a non \texttt{0} or \texttt{1} value is unintentionally returned. This acts
to prevent unnecessary errors. This is possible as OCaml has a strict type system.

The type \textbf{char} is single characters, for example \texttt{'a'}, \texttt{'!'} and \texttt{'E'}.

Mathematical operators only act between expressions of type \textbf{int}. These are \texttt{+ - * / mod}.
The result of such an expression will also have type \textbf{int} (remember this when using division operations).

Comparison operators compare expressions of the same type. This means that Booleans and characters can
also be compared. These are \texttt{= < <= > >= <>} (where \texttt{<>}
is the operator for "not equal to" in OCaml). The result of such an expression will have type \textbf{bool}.

Boolean operators \texttt{\&\&} and \texttt{||} are "and" and "or" respectively and compare expressions
of type \textbf{bool}.

Operators of higher precedence are evaluated first. For example, \texttt{*} has precedence over \texttt{+}
and "and" has precedence over "or". Otherwise, operators are evaluated from left to right.

\textit{If statements} have the following syntax: \texttt{\textbf{if} \textit{expression1} \textbf{then}
\textit{expression2} \textbf{else} \textit{expression3}} where \textit{expression1} has type \textbf{bool}, and
\textit{expression2} and \textit{expression3} have the same type.

\section{Names and Functions}
One can assign a value to a name using the syntax: \texttt{let \textit{name} = \text{expression}}. This can
be extended to compound expressions using the syntax: \texttt{let \textit{name1} = \textit{expression1} in let \textit{name1}
= \textit{expression2} in ...}. For example:
\begin{lstlisting}
    # let x = 200 in x * x * x;;
    # let a = 500 in (let b = a * a in a + b);;
\end{lstlisting}. The rightmost expression will be fully evaluated first, even if brackets aren't explicitly
written.

A \textit{function} takes some input, called an \textit{arguement}, and evaluates to some value. These are
created using the syntax: \texttt{let \textit{name arguement1 arguement2} ... = \textit{expression}}.
This function has type $\alpha \rightarrow \beta$, $\alpha \rightarrow \beta \rightarrow \gamma$ etc. for
some types $\alpha$, $\beta$, $\gamma$, where the rightmost type corresponds to the type of the value of
the function and the other types correspond to those of the arguements respectively. For example:
\begin{lstlisting}
    # let cube x = x * x * x
    val cube : int -> int = <fun>
\end{lstlisting}
where the second line shows the output by the OCaml interpreter, making the type of the function explicit.

In OCaml, whitespace (notably indentation) is used purely to aid readability.


\end{document}
