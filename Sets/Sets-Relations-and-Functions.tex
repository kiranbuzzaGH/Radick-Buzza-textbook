\documentclass[12pt]{article}
\usepackage{amsmath}
\usepackage{amsthm}
\usepackage{amssymb}
\usepackage[left=2cm,top=1cm,right=3cm,bottom=1cm]{geometry}
\usepackage{graphicx}
\usepackage{listings}

\newtheorem{axiom}{Axiom}
\newtheorem{remark}{Remark}


\begin{document}

\begin{center}\LARGE\bf
    Sets, Relations and Functions
\end{center}

\section{The Axiom of Extension}
Sets (or collections) have elements (or members). Any object (concrete or otherwise), may be an element
of a set. This extends to sets, which may be an element of some other set. In fact, it is generally sufficient
to only discuss sets of sets, as the properties will extend to all other sets. There are some ways to relate sets: belonging, inclusion and equality.

If an object $x$ is an element of set $A$, we say that it \textit{belongs} to $A$, or "$x$ is contained
in $A$". Denote this as $x$ $\epsilon$ $A$.

If a set $A$ is \textit{included} in some set $B$, then $A$ is a subset of $B$. Denote this as $A \subset B$ or
$B \supset A$. Some related definitions: set inclusion is \textit{reflexive} since a set is included in
itself ($A \subset A$)\footnote{Clearly, $A \subset B$ is the same as $A \subseteq B$ in this notation.}; set inclusion is \textit{transistive} as $A \subset B$ and $B \subset C$ implies
$A \subset C$. (Conversely, belonging is neither reflexive or transistive).

Sets can also be related by equality ($A = B$), as defined by the Axiom of Extension.
\begin{axiom}[Axiom of Extension]
    Two sets are equal if and only if they have the same elements.
\end{axiom}
In terms of  inclusion, this can be rewritten as "$A \subset B$ and $B \subset A \Leftrightarrow A = B$".

A set is said to be a \textit{proper} subset of another if $A \subset B$ and $A \neq B$; that is, all
subsets are proper except from the set itslf, which is a subset but not proper.

Equality is \textit{symmetric}, that is $A = B \Leftrightarrow B = A$, whereas set inclusion is \textit{antisymmetric}
(that is, $A \subset B$ and $B \subset A \Rightarrow A = B$)\footnote{Antisymmetry is defined by "if $aRb$ and $bRa$, then $a = b$" for
some binary relation R.}.

\section{The Axiom of Specification}
The basis of most Set Theoretic principles is to construct new sets from old sets. The most important such principle
is the so called "\textit{Aussonderungsaxiom}":
\begin{axiom}[Axiom of Specification]
    To every set A and every condition S(x) there correponds a set B whose elements are exactly those
    elements x of A for which S(x) holds.
\end{axiom}
That is to say, anything one can say about a the elements of a set defines a subset. The condition must form a
valid \textit{sentence}. A sentence is formed using the atomic sentences of belonging ($x$ $\epsilon$ $A$)
and equality ($A = B$), which are used to create more specific sentences using the following logical operators:
\begin{itemize}
    \item and
    \item or
    \item if and only if
    \item not
    \item if - then - (equivalently \textit{implies})
    \item for some (equivalently \textit{there exists})
    \item for all.
\end{itemize}
There are a few general rules for sentence construction\footnote{Some general notes: the set of a single object is not the same as that object itself;
"for some $y$ ($x$ $\epsilon$ $A$)" is equivalent to "$x$ $\epsilon$ $A$"; "for some $x$ ($x$ $\epsilon$ $A$)" and
"for some $y$ ($y$ $\epsilon$ $A$)" are equivalent.}:
\begin{enumerate}
    \item \textit{and}, \textit{or} and \textit{if and only if} are placed between two sentences
    \item \textit{not} can be placed before a sentence enclosed by parentheses
    \item the dashes in \textit{if - then - } are replaced with sentences
    \item follow \textit{for some} or \textit{for all} by a letter, in turn followed by a sentence
\end{enumerate}

We can write the Axiom of Specification symbolically as: $B = \{x \text{ }\epsilon \text{ }A: S(x)\}$. This set $B$ is uniquely
specified due to the Axiom of Extension.

\begin{remark}[Russell paradox]
    Nothing contains everything (or equivalently, there is no universe)
\end{remark}
\begin{proof}
    Assume there is a set $A$ which contains everything.

    Consider the condition: $\text{not } (x \text{ } \epsilon \text{ } x)$. Let us rewrite this as ($x$ $\epsilon'$ $x$) for
    ease of use.

    Constructing the set $B$ from $A$ where this condition holds:
    \begin{displaymath}
        B = \{x \text{ } \epsilon \text{ } A: x \text{ } \epsilon' \text{ } x\}
    \end{displaymath}

    Following from the Axiom of Extension,
    \begin{displaymath}
        y \text{ } \epsilon \text{ } B \text{ if and only if } (y \text{ } \epsilon \text{ } A \text{ and } y \text{ } \epsilon' \text{ } y)
    \end{displaymath}

    For $A$ to contain everything, this must hold for any $y$. Consider $y = B$.
    If $B$ $\epsilon$ $A$, then $B$ $\epsilon$ $B \Leftrightarrow B$ $\epsilon'$ $B$. This is clearly a
    contradiction so $A$ does not contain $B$ and by extension, $A$ cannot contain everything.
\end{proof}

The Russell Paradox gives us an example of a condition that produces an illegal set. Some texts name these
illegal sets as "classes".

\section{Unordered Pairs}
In order to make any progress it is necessary for us to make an assumption: \textit{there exists a set}.
This will be formulated more carefully later on but for now this is sufficient.

Given the assumption that there exists a set, one can easily construct the \textit{empty set} using the
Axiom of Specification, using the condition $x \neq x$. Clearly there are no
elements of such a set. This set is unique by the Axiom of Extension. Denote this as $\{x \text{ } \epsilon \text{ } A: x \neq x \} = \varnothing$.

For any set $A$, $\varnothing \subset A$ as every element in $\varnothing$ is in $A$. This is a vacuous truth.\footnote{The
arguement can also be made from the other direction: the only case where $\varnothing \subset A$ is false
is where there is some element in $\varnothing$ that is not contained in $A$. Since $\varnothing$ is empty, this
doesn't make any sense and so the statement can never be false.}

\begin{axiom}[Axiom of Pairing]
    For any two sets there exists a set that they both belong to.
\end{axiom}
That is, for some sets $a$ and $b$, there exists a set $A$ such that $a \text{ } \epsilon \text{ } A \text{ and } b \text{ }
\epsilon \text{ } A$.

Consider the set that contains $a$ and $b$ and nothing else\footnote{The existance of such a set is actually equivalent to the Axiom of Pairing:
if there is some set $A$ such that $a \; \epsilon \; A$ and $b \; \epsilon \; A$, then we can construct
a set with only $a$ and $b$ in using the sentence "$x = a \text{ or } x = b$".}.
This can be constructed from the Axiom of Specification using the condition $S(x) =
"x = a \text{ or } x = b"$. By the Axiom of Extension, this set is unique. We call this the \textit{pair}
or, more specifically, the \textit{unordered pair} formed by $a$ and $b$. Denote this set as
\begin{equation}
    \{x \; \epsilon \; A: x = a \text{ or } x = b\} = \{a, b\}.
\end{equation}
Note that this isn't the same as the set of all elements in $a$ or $b$ but the set containing only two
elements: $a$ and $b$.

The \textit{singleton} of $a$ is a special case unordered pair $\{a, a\}$, denoted by $\{a\}$, where the set has $a$
as its only element. $a \; \epsilon \; A$ is equivalent to $\{a\} \subset A$.

Considering the unordered pair of any set and the empty set, it follows from the Axiom of Pairing
that every set is an element of some other set. We can also infer the existance of infinitely many sets by construction:
first consider the singletons $\varnothing, \{\varnothing\}, \{\{\varnothing\}\}$ etc.; now consider the pairs of such sets, and the
pairs of these and the singletons and so on. Note that all of these sets are unique by the Axiom of Extension.

If $S(x)$ is a condition on $x$, then the set of all $x$ that satisfy this condition is given by $\{x:S(x)\}$,
granted that this is a valid set. Rewriting the generic set constructed by the Axiom of Specification,
\begin{displaymath}
    \{x\;\epsilon\;A: S(x)\} = \{x: x\;\epsilon\;A \text{ and } S(x)\}.
\end{displaymath}

\section{Unions and Intersections}



\end{document}
