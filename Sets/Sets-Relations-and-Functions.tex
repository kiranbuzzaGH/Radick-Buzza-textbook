\documentclass[12pt]{article}
\usepackage{amsmath}
\usepackage{amsthm}
\usepackage[left=2cm,top=1cm,right=3cm,bottom=1cm]{geometry}
\usepackage{graphicx}
\usepackage{listings}

\newtheorem{axiom}{Axiom}


\begin{document}

\begin{center}\LARGE\bf
    Sets, Relations and Functions
\end{center}

\section{The Axiom of Extension}
Sets (or collections) have elements (or members). Any object (concrete or otherwise), may be an element
of a set. This extends to sets, which may be an element of some other set. In fact, it is generally sufficient
to only discuss sets of sets, as the properties will extend to all other sets. There are some ways to relate sets: belonging, inclusion and equality.

If an object $x$ is an element of set $A$, we say that it \textit{belongs} to $A$, or "$x$ is contained
in $A$". Denote this as $x$ $\epsilon$ $A$.

If a set $A$ is \textit{included} in some set $B$, then $A$ is a subset of $B$. Denote this as $A \subset B$ or
$B \supset A$. Some related definitions: set inclusion is \textit{reflexive} since a set is included in
itself ($A \subset A$); set inclusion is \textit{transistive} as $A \subset B$ and $B \subset C$ implies
$A \subset C$. (Conversely, belonging is neither reflexive or transistive).

Sets can also be related by equality ($A = B$), as defined by the Axiom of Extension.
\begin{axiom}[Axiom of Extension]
    Two sets are equal if and only if they have the same elements.
\end{axiom}
In terms of  inclusion, this can be rewritten as "$A \subset B$ and $B \subset A \Leftrightarrow A = B$".

A set is said to be a \textit{proper} subset of another if $A \subset B$ and $A \neq B$, that is, all
subsets are proper except from the set itslf, which is a subset but not proper.

Equality is \textit{symmetric}, that is $A = B \Leftrightarrow B = A$, whereas set inclusion is \textit{antisymmetric}
(that is, $A \subset B$ and $B \subset A \Rightarrow A = B$).






\end{document}
